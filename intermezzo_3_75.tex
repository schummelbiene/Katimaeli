
\subsection{Der Hochkönig}
\label{sec:der-hochkonig}
Um keine Komplikationen mit dem Zeitgefüge zu bekommen, werden hier keine genauen Daten angegeben. Die Sequenz findet vor unserem Zusammentreffen 1018 BF in Gareth statt 

(\gold{+ 100 D Leibrente})


In Gareth: Ich treffe Karypto Extraordinarius und Sonnhard. Wir plaudern über alte Zeiten und das, was im letzten Jahr geschehen ist.

Beispielsweise wird Galotta in Tobrien (später schwarze Lande) vermutet. Außerdem überlegt der alte \name {Hochkönig Arombolosch Sohn des Agam} in Morolosch im Amboss, zur Neuwahl aufzurufen, da er schlechte Träume über seine Untreue gegenüber seinem Volk hat. Deshalb will Sonnhard nach Morolosch reisen. 

Am nächsten Tag gehen wir zum Hesindetempel, um dort Informationen einzuholen: Sie als Informationszentrale über die Aktivitäten Borbarads wissen von 'Borbaradianern' in Fasar. Zudem wurde ein Stoßtrupp Weißmagier in die Gor geschickt, und ein Kloster im Khoremgebirge (zwischen Gor und Khom) wurde untergraben. Außerdem informieren wir uns über unsere Reiseroute. 

Wir reisen nach Murolosch. Die Stadt ist in Fels gearbeitet, die Gänge sind ungefähr drei Meter hoch. Fackeln sind an den Wänden. Die Zwergenhöhlen sind in den Stein gehauen. Wir finden zu der großen Markthalle. 
\gold{+1 Bonus-AP für Stift geben}

Nach einem Gespräch auf Rogolan mit der Wache des Königspalastes treffen wir den Hochkönig \name{Arombolosch, Sohn des Agam}. Er empfängt uns, da wir 'die Gezeichneten' sind. Wir reden über die Geschehenisse für das Land. Ein bisschen Zwergisch: 'Burbra Domron' heißt 'schwarzer Borbarad' auf Rogolan.  Malmarok ist der symbolischem, heilige Hammer des Hochkönigs. Der Hochkönig der Zwerge beschließt während unseres Gespräches, zur Neuwahl aufzurufen. Angarok Rognarok heißt 'Hochkönig'. Wir sollen Briefe zu den Bergkönigen bringen, um sie zu fragen, ob sie sich zur Wahl aufstellen lassen wollen oder jemanden anderen vorschlagen. Außerdem sollen wir den Malmarok aus Ferdok holen und hierher bringen. Wir bekommen dazu einen Siegelring von ihm. \name{Growin Sohn des Gorbosch} wacht über den Hammer, er ist Schankwirt. 
Die Bergkönige und ihre Reaktionen sind
\begin{enumerate}
\item \name{Gillemon Sohn des Gillim} (nördlicher Kosch, konservativ, grimmig) kandidiert nicht, denn er hat wichtigeres zu tun, als sich für eine Wahl aufzublasen, keinen Vorschlag;
\item \name{Nirwulf, Sohn des Negromon}, Angbar, lustig, oioioi, oberster Richter, keine Antwort;
\item \name{Fargol, Sohn des Fanderam}, Eisenwald, er kandidiert;
\item \name{Tschubaks Sohn des Tuagel},  Xorlosch, bartlos, Wutattacke, glaubt, dass er ein Recht auf den Titel des Hochkönigs hat, weil er aus Xorlosch kommt;
\item \bf{\name{Omgrasch, Sohn des Orbal}}\normalfont, in Brilliantzwerg in den Beilunker Bergen, 'endlich', übereifrig, kandidiert;
\item \name{Garbalon, Sohn des Gerambalosch} im Finsterkamm in der Finsterkloppenbinge, kein Kandidat, da er eigene Probleme hat. 
\end{enumerate}

Anschließend kommen wir nach Ferdok zum Brauereiausschank. In einem Nebenraum hängt der Hammer. Er ist ungefähr 2 Spann groß und reichlich verziert, doch leider eine Fälschung!! Growin bricht in Panik aus und wir erklären uns bereit, bei der Suche und Aufklärung zu helfen.  

Folgende Personen sind verdächtig, da sie in den letzten Stunden im Schankraum waren:
\begin{itemize}
\item eine Menschenfrau mit roten, hüftlangen Haaren;
\item zernarbtes Gesicht, dunkelblauer Umhang, dunkelblond, Tulamide, der 'Silberkrug' gemurmelt hat;
\item drei Xorloscher Zwergenkrieger, die vor fünf Stunden hier waren.  
\end{itemize}
Wir gehen zum 'Silberkrug'. Ich besteche die Schankdame\gold{-1 D Bestechung}. Sie erzählt uns, dass er bis vorhin da war. Er war recht schweigsam, hat ihr aber erzählt, er wolle Ferdok gen Westen verlassen.  

Ich durchsuche sein Zimmer und finde einen Zettel, auf dem, wie sich später durch Karypto herausstellt, ein 'b' in Zhayadrunen steht. Wir verfolgen den Mann nach Westen und finden ihn zusammen mit drei Holzfällern an einem Rastplatz. Er bietet uns an, gemeinsam Mittag zu essen. 

Sie wollen nach dem Essen wieder die Arbeit aufnehmen. Wir fordern sie mit dem Zwegensiegel auf, uns ihre Habseligkeiten zu zeigen. Es kommt zu einem Kampf. Sonnhard wird verwundet.

Ich erringe im Gefecht den Hammer, aber sie geben nicht auf. Als sie bemerkten, dass sie keine Chanche haben (?),  sagt der Tulamide
\mybox{7}{Borbarad braucht den Hammer. Er wird ihn früher oder später bekommen.}
Dann teleportieren sie sich weg, indem sie sich an ihre Gürtel greifen. 

Wir bringen den Hammer zurück. Auf dem Weg will Karypto ihn magisch untersuchen, was nicht gelingt und in einer astralen Entladung in Form von Glitzer und Regenbögen endet. 

Im Ferdoker Bierschank ist uns Graf Growin, Sohn des Gorbosch, so dankbar, dass wir Freibier auf Lebenszeit erhalten werden. Er bietet uns 20 Axtträger als Begleitung nach Morolosch an, die wir gerne annehmen.  

Wir erfahren noch, dass die Kirchenspaltung der Praioten beendet ist. Iberian erkennt seinen Fehler, dass er doch nicht das neue Oberhaupt ist. Er bereut seine Taten. 
\mybox{7}{\gold{1 Bonus-AP und *Singen für Firuns-Tannen-Bemühungen}}

%%% Local Variables: 
%%% mode: latex
%%% TeX-master: "borb_head"
%%% End: 
