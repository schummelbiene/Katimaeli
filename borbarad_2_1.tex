\section{Unsterbliche Gier}
Ich war bei meinem Stamm. 
\paragraph{Danae, Sonnhard Mauernbrecher, Sonea, Karypto}
\underline{Spätherbst, 1016 BF}\\
Ein mittelreichischer Reiter übergibt mir einen ledernen Brief, gutes Papier Bär auf grünem Grund. Der ist von \name{Herzog Waldemar von Weiden}, ich soll nach Trallop kommen, um dort meinen Schwertarm zur Verfügung zu stellen. 

Da möchte ich gleich mit dem Reiter mitkommen. 

\underline{1. Boron 1016 BF, Winter}
\gold{+10 Dukaten}
Ich muss auf die Bärenburg. Auf dem Marktplatz treffen wir uns und gehen zusammen hoch. Die Burg ist riesig. 

\name{Walpurga von Weiden}, Tochter des Herzogs und ihr Gemahl \name{Dietrad von Ehrenstein} (tobrischer Prinz) sind da, ebenso Herzogin \name{...}

Trainingskampf mit Zweihänder in Hof (\name{Rondrian}) \gold{Zweihänder *}

Abends: Wir sitzen im Thronsaal, Herzog und Herzögin(\name{Frau Girlande von Weiden}) und wir sind da. Wir erzählen von unseren Heldentaten.

Aberglaube in Weiden: Man soll das Namenlose nicht benennen, denn sonst holt man es ins Haus.

Es verschwinden Leute. Drei Dutzend Menschen sind schon verschwunden. Unheiliges. In Baliho besonders viele. Einfache Bevölkerung. 3 - 4 Monate. Herzogin von Weiden: Westen von Weiden. Archeburg: Da ist eine alte Burg, da spukt es, da können wir anfangen. $<3 <3 <3$ $\varepsilon>0$ \name{Walmir von Riebbshoff} ist der Besitzer, der immer noch durch Weiden spuken soll. 

\gold{+ Borndolch, + Wurfdolch mit dreikantiger Klinge + Achat mit Zeichen von Boron und Hesinde}(sollen gegen Zauber schützen), Bärenfellmütze, usw.

Draußen: Kutsche mit Kufen (Kaleschka) unten dran. \name{Boril Bargoltin} ist unser Kutscher.


\subsection{Sonnhards sinnierendes Selbstgedusel}
Vorläufiges Ziel: Balsaith, dann zum Kaff 'Scheuzen', dann nicht mehr weit zur Burg (wie auch immer die hieß).

Reise verläuft angenehm, ist jedoch nicht sehr schnell.
\underline{2. Boron.}
Begegnung: Mann, welcher Streitross mitführt. Waffendecke zieren zwei Schwerter ('Antworter und Vergelter') , es scheint sich um \name{Raitri Conchobair} zu handeln. Er kommt aus dem Bornland, hat sich jedoch streckenweise verlaufen. Er begibt sich weiter seines Weges.

Wir erreichen eine Gaststube; Karypto spürt eine Zerwürfnis in der Macht, wir ignorieren seine Bedenken.

Es geht das Gerücht, ein weißer Drache würde in der Umgebung sein Unwesen treiben.
Neben einem Handwerker, sitzt ein auffällig unauffälliger Mann. Sein Name ist \name{Fenriel}. Karypto und er, unterhalten sich kurz und unerbittlich.

Karypto hält uns einen Vortrag zu seinem Auge; es/\name{ER} spricht zu ihm. Er leidet deswegen an einer Art Verfolgungswahn. \name{ER} kann seine (Karyptos) Gedanken lesen.

\underline{Der nächste Tag (3. Boron):}
Sonea hatte einen Alptraum und in ihr Zimmer wurde eingebrochen, jedoch nichts entwendet. Sie fand Neuschnee und Fußspuren auf ihren Fensterbrettern. Wir reisen weiter. Des mittags erreichen wir Balsaith. Hier wohnt der Baron der Baronie Brachfelde. Wir reiten zu ihm. Er ist der Schmied des Dorfes. Er beschreibt uns den Weg zur Burg: Nach Scheuzen und dann einen kleinen Pfad zur Burg. Er schmiedet gerade ein Silberschwert für \name{Mardulf von Hartsteen}.


Nach diesem kurzen Intermezzo reisen wir weiter nach Schneuzen. Das Dorf zählt 50 Einwohner und ist die letzte Station vor der Archeburg. Wir begeben uns zur Ruhe. In der Nacht werden wir von Sonea geweckt, welche wieder Fußspuren auf dem Fensterbrett des offenen Fensters gefunden hat. Vor dem Fenster befinden sich weitere Fußstapfen. Sonea folgt den Fußstapfen, sie verläßt jedoch bald der Mut; die Fußstapfen verdoppeln sich nach einiger Zeit - Hand und Fußabdrücke sind zu sehen. 


\underline{Am nächsten Morgen (4. Boron):}
Wir machen uns auf den Weg zur Burg, verlaufen uns aber ein bisschen. Auf dem Rückweg finden wir einen toten, enthaupteten Körper. Er scheint Jäger zu sein. Sein Kopf liegt verdächtig weit vom Körper entfernt, ohne das Blut sich auf dem Verbindungsweg zwischen Kopf und Hals sich befinden. Einige Zeit später finden wir, auf dem richtigen Weg, dann doch die Burg.
Karypto sieht überall Spuren der Magie in der Burg verteilt. Ebenso haben Füße Spuren hinterlassen, diese lassen sich jedoch nicht eindeutig zuordnen, sie scheinen sich in Richtung der adligen Gemächer zu bewegen, oder eher, bewegt zu haben. Wir finden auf dem Weg dahin einen Boronsanger. Die Spuren führen zu einer einzelnen, eingestürzten Krypta. Wir finden einzelne skeletare Bestandteile von Menschen. An dem Ort, was wir aus Haupthaus ausmachen, findet Sonea Spuren eines löwenartigen Wesens. Aus nicht sonders erklärungsreichen Gründen steigen wir eine Treppe hinab - in die Ruine eines ehemaligen Turm. Sie scheint schon seit hunderten von Jahren bewohnt zu sein, es befinden sich intakte Bilder und eine alte und trotzdem aktuelle Bibliothek darin. Wir machen uns unbefriedigt über die geringen Erkenntnisse auf den Weg zurück. Wir kehren wieder ins Dorf zurück und legen uns Schlafen mit dem Vorsatz, Nachtwachen zu halten.

\subsection{Katimälis chaotische Kritzeleien}
\underline{5. Boron} Am frühen Mittag sind wir wieder bei der Archeburg an.
\begin{itemize}
\item Karypto hat einen Schatten gesehen, der durch Wände huscht
\item es gibt einen Elfenschamanen, der Orks angeheuert hat, um den Leichnam von vermutlich \emph{von Riebeshoff} aus seinem Sarg zu holen
\item diese haben dann auch die Zeichen von Praios und Boron zerstört, die in die Kryptatür von \emph{von Riebeshoff} eingemeißelt waren 
\item die Ruine war so eingerichtet, als ob da noch jemand wohnt
\item in einer der vier Krypten ist ein Leichenstapel aus drei Menschen, es gibt zwei Paare, die je in einem Abstand eines halben Jahres gestorben, der Stapel ist aufsteigend geordnet.  
\end{itemize}

\gold{+1 Bonus-AP}

\underline{6. Boron}

-1 LeP

\name{Fredo} versteckt sich, wurde beim Holzhacken angegriffen von löwengleichen, ochsengroßen, dreizüngigen, geflügelten, schwarz-violetten Wesen angegriffen. Seit 4-5 Monden hat er 

Familie von Fredo: \name{Bregelsorm}: Waldrand, dann schmaler Weg

Familie ist tot?!

Das Böse ist nicht mehr auf der Archeburg. 
Wir kommen tiefnachts im Gasthaus an.

\stats{volle LeP}

\underline{7. Boron}

Wir reisen nachher nach Anderrath über Trallop.

EIN VERDAMMTER EISDRACHE ERSCHEINT AM HIMMEL!

Sonnhard versteckt sich im Schnee. Katimäli denkt an die letzte Begegnung mit einem Eisdrachen.