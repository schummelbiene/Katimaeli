\section{Pforte des Grauens}
\label{sec:pforten-des-grauens}
\paragraph{Sonnhard, Sonea, Karypto, Mentorin Varna Henanduria}
\paragraph{Weltgeschehen}
\label{sec:weltgeschehen}

\begin{enumerate}
\item Von Maraskan gab es im Boron eine Massenflucht von 2000 Personen, die sich als Nachfahren aranischer Auswanderer auf ihre Heimat zurückbesinnen. Sie kamen in Zorgan an Land, und sind dann auf dem Pilgerweg nach Baburin verschwunden. Der Anführer heißt \name{Endijian}. 
\item Die Stadt Alta\"ia auf Altoum ist 'von einer fremden Macht' vernichtet worden. Gestohlen wurde die 'leuchtende Kugel der Hesinde'. Alle sind tot, Tempel und Gebäude sind vernichtet.  
\item Hakuna Matassa wurde in Fasar gefasst, weil er mit dem finstere Machenschaften nachging (Borbarad?)
\item Verschiedene Personen verkünden die Rückkehr Borbarads. 
\item Kirchenspaltung der Praioskirche ist zu Ende.
\end{enumerate}

Im garether Gasthaus 'Roter Hahn' findet eine Verlosung von Gegenständen der Hesindekirche statt. Ich kaufe zwei Lose im Wert von \gold{-20 D},
und gewinne nix. Ich treffe meine alten Gefährten dort wieder. 

Sonea ist ihrer Tätigkeit als Kopfgeldjägerin nachgegangen, im Khoram-Gebirge steht ein Kloster des Borbarad. Sie hat sich eingeschmuggelt. Oberster ist der 'Bettelmönch'. Expedition in die Gor, um Artefakte Borbarads zu bergen, Überfall auf die Expedition seitens der KGIA. Alle sind unschädlich gemacht worden, außer der Bettelmönch. 

Können Anhänger Borbarads bisher nur eingeschränkt Magie wirken?

Tuzak ist die Hauptstadt von Maraskan. 

Ein Beilunker Reiter kommt sehr erschöpf zu uns in den Nebenraum, stellt uns einige Kontrollfragen, und übergibt uns einen dreckigen Brief von der KGIA. 

\paragraph{Brief}
\label{sec:brief}
Sie beobachten uns seit zwei Jahren. Er ist auf Maraskan, und seine Forschungen sind eventuell für uns wichtig. Sondermission, um eine Verschwörung zwischen Echsenpack, aufständige Maraskaner und Khunchomer Schmuggler aufzudecken. 
Echsischen Ruinen in Vimbatia am Osthang des Amdeggyn-Massivs wurden untersucht in Begleitung des brabaker Questadors und guten Maraskankenners \name{Borotin Almachios}. Personen, die mit dem Echsen unheimlich aufgefallen sind. Von Storebrandt finanziert hält sich \name{Hilbert Puspereken} dort(?) auf, der Ausgrabungen vornimmt, dessen Gesinnung nicht klar ist. Almachios erzählt von der mysteriösen Echsenstadt Akrabaal irgendwo im Dschungel. Der Seeschlangenfriedhof wurde gefunden und Höhlen. Die Personen der Verschwörung wurden von \name{Rahjo Brabaker}, dem 'Auswählten', kahlköpfiger Novadi. 

Er ist auf dem Weg nach Tuzak. Insel scheint ihm als eine einzige tödliche Falle. Sein Plan: Fürst \name{Herdin} berichten. Wir sollen uns bei ihm Melden. Alternativ sollen wir uns an KGIA-Oberst \name{Praoitin von Rallerau} wenden. 

Der Absender ist Delian von Widbrück. Mitte der Phex 24 Hal (1017 BF). 

\paragraph{Weitere Pläne und Recherechen}
\label{sec:weitere-plane}
Karypto schlägt vor, nach Mhanadistan/Maraskan/Tuzak zu reisen. 

Varna kümmert sich um eine Maraskankarte. 

\begin{figure*}[ht!]
  \centering
  \includegraphics[scale=0.7]{maraskankarte.jpg}
  \caption{Maraskankarte}
  % \label{fig:maraskankarte}
\end{figure*}

Nandusprophezeiung: Butler Jordan fragt ob Zentauren Fußvolk oder Reiterei ist. Danach entrückter Zustand: 
\mybox{10}{Wenn vier Hörner und acht Beine die Wasser teilen und die Bäume verbünden, \\
werden sich erschöpfen die Wasser und begraben die Lande vor den Teilenden.}

Blinde Boron-Novizin \name{Solva} fordert uns auf, nach Punin zu reisen. 'Eure Erhabenheit' ist die Bezeichnung des schwarzen Raben des Al'Anfaner, 'eure große dunkle Schweigsamkeit' ist die Anrede für das Puniner Oberhaupt.

Ich bin sauer auf Varna Henanduria und will es ihr noch heimzahlen, weil sie immens arrogant ist. Unser Magus erkrankt. 

Für ein geeignetes Hauptquartier brauchen wir 2000 Dukaten. Wir brechen auf. 

\paragraph{17. Rahja 1018 BF}
\label{sec:mitte-rahja-1018}
Wir erreichen Punin. Ausgelassene Stimmung, 'Spaß auf den letzten Drücker', niemand will an Morgen denken. Wir gehen zum Borontempel, selbst auf dem Platz vornedran ist Schweigen. 
Wir werden zum Oberhaupt der Puniner Boronkirche durch den verschlungenen Tempel geführt. Es gibt Berichte über kirchliche Schwerter, Endurium wurde auf Maraskan wiederholt auf dem Weg von der Miene \name{Aram Anchy} nach Tuzak gestohlen. Rebellengruppe \name{Haranydad} hat das erste Mal die Miene überfallen. Der Verantwortliche des Enduriumtransportes ist ein Banner der Drachen-/Adlergarde. 

Wir werden gefragt, ob wir bereit sind, nach Maraskan zu reisen. Alle außer mir willigen ein. Ich werde hinausgeleitet.

\mybox{15}{
Ich werde nicht aus freien Stücken wieder nach Maraskan gehen. Die Insel ist verflucht, ich werde dort verfolgt. Die Überfahrt alleine ist ein großes Risiko. Ich bin auf einem Schiff hilflos den Meer ausgeliefert, Piraten, Meerungeheuern. Auf Maraskan selbst bin ich gefangen, es gibt tausend Gefahren, die ich nicht vermute. Ich kenne die Tiere und Pflanzen nicht. Ich kenne die Menschen und das Gebirge nicht. Ich müsste Eno alleine zurücklassen, die Überfahrt schafft sie nicht noch einmal. Und das, weil mich die Zwölf-Götter-Kirche dazu bittet? Meine Gefährten kennen die Gefahren nicht. Sie sind großmütig, weil sie das Ausmaß und den Schrecken unterschätzen. Warscheinlich weis nicht einmal der Mann, der mich dorthin schickt, welchen Gefahren er mich aussetzt, wenn er mich auf die Insel schickt.     
}

Meinen Gefährten möchte ich sagen:
\begin{enumerate}
\item Ihr unterschätzt die Gefahren. Es gibt keinen Grund, den Schauererzählungen über Seeschlangen, algiges, versumpftes Meer, Charypteroth-Dämonen und dergleichen nicht zu glauben.
\item Wir alle haben schon viel erlebt, aber Maraskan ist eine andere Klasse von Gefahr!
\item Eine Gefahr wie der Dämonenkönig handelt planmäßig. Gefahren wie die See, Stürme, Naturgewalten tun das nicht!
\item Maraskan gleicht für mich einem Gefängnis.
\item Ich kann Eno nicht mitnehmen.
\item Aufgrund meiner letzten Durchquerung der Seeblockade werde ich im Perlenmeer gesucht.
\end{enumerate}
Meine Gefährten wollen mich überzeugen, doch mitzukommen. Sie unterschätzen nach wie vor die Gefahren. 
\paragraph{18. - 30. Rahja 1018 BF}
Ich lasse Eno frei und komme mit. Wir reisen nach Samra, wo wir die namenlosen Tage verbringen. Ruinen von Zhamorrah sind dort, legendäre Ruine der Stadt Zhamorrah östlich von Samra. 

\paragraph{1. - 4. Praios 1019 BF}
(\gold{+ 100 Dukaten Leibrente})
Nach Khunchom. Neunflüssige Mündung des Mhanadi in die See. Es gibt ein Maraskan-Viertel. Informationen
\begin{enumerate}
\item Gruppierungen der Aufständischen: Haranida (Enduriumüberfall), der Diskus von Boran;
\item Übergriffe von Maraskanexilanten auf Festländer;
\item Fürst Herdin regiert tyrannisch, Delian von Widbrück ist nach wie vor sein Berater; 
\end{enumerate}
Im Borontempel 'Tempel des Todes': Eine sechsfingerige Hand aus schwarzem Nebel erscheint und versucht nach uns zu greifen. Sie verschwindet, kurz bevor sie uns erreicht. Die Geweihten haben sie nicht gesehen. Capitain \name{Haimund ibn Mhukkadin} wird uns auf dem Schiff 'Perlbeißer' am Ankerplatz 11 am Folgetag unseres Erscheinens auslaufen bei Nacht. Maraskaner tragen bunt.

\paragraph{5. Praios 1019 BF}
Wir besuchen die Dracheneiakademie und der Vorsteher empfängt uns. Er will uns Endurium abkaufen, wenn wir welches ergattern können.

Meine Maraskankleidung hat die Farben schwarz, grün, ungefärbt. \name{Yalina} ist eine Matrosin, wir fahren auf einer Zedracke, kiellos. Bewaffnetes Schiff, pro Seite eine Hornisse. Wir werden zu Capitain Haimud ibn Mhukkadin geführt. Einfache Kleidung, lange schwarze Haare. In einer Kiste, die uns die Boronkirche zur Verfügung gestellt hat, sind:
\begin{enumerate}
\item einige Haumesser, 
\item drei Sturmlaternen,
\item taubeneigroßer Gwen-Petryl-Stein,
\item zwei Fläschchen Antidot,
\item eine Flasche Einbeerentrank ($\sim$ 40 LeP),
\item vier Portionen Stinkmirbel (Stinkezeug, dass gegen Ungeziefer gut ist),
\item 20 Beutel 12-Blatt-Tee (heiß genossen beugt er der Übertragung von Krankheiten vor), 
\item eine Portion Olgenwurz-Absud (präventiv, einen Tag vor Wirkungsbeginn trinken, stärker als 12-Blatt-Tee),
\item drei Portionen Kugris (Waffengift, Muskelschmerzen).
\end{enumerate}

\paragraph{Überfahrt}
\label{sec:uberfahrt}

Kontrollschiff wird durch Illusion unseres Magus abgelenkt.

Abends erscheinen enorm scharze Wolken. Ein Tropensturm zieht auf, tritt oft im Zusammenhang mit einem Rondrikan auf. Der Sturm tobt.

\stats{ - 10 LeP, 1 Wunde}, da von Fass überrollt. Wir erreichen die maraskanische Küste. Unser Schiff ist so beschädigt, dass man es nicht mehr steuern kann. \stats{-4 LeP} von den Folgen des Sturms, Rest \stats{18 LeP}.

\paragraph{7. - 8. Praios 1019 BF}
\label{sec:7.-praios-1019}
21 Überlebende, 2 Tote, 9 Schwerverletzte, Käpten, wir fünf. Das Schiff ist kaputt. Wir sind am Strand, nach fünf Schriff fängt der Dschungel an. Fingerlange Ameisen krabbeln über uns. Varna ist schwer verletzt.

130 - 150 Meilen nördlich von Tuzak. Wir bauen ein Lager auf. Vorräte für 2 Tage Wasser, 5 Tage Essen sind noch zu retten. Es fängt abends an zu regnen in Strömen. Wir sammeln Trinkwasser.

Nachts 2 Duzend Rebellen, Anführerin will den Capitain sprechend. Sie heißt \name{Enjisab}. Der Stamm sind die 'Wöpfeltiger'. Yalina erklärt uns, dass wir unbewaffnet mit ihnen in ihr Lager gehen sollen. 'Der Colonel will mit uns reden', sein Name ist \name{Orsino von Ragath}.

Wir werden in den Dschungel geführt. Lichtung. Strickleitern fallen von den Bäumen, dort sind Häuser gebaut. Wir werden in das größte, prunkvollste Gebäude geführt. Wir haben das Lager erreicht. Wir treffen dort den Colonel \name{Orsijin von Hiera}. Er behandelt uns irgendwie respektvoll, aber lässt auch seine Macht spielen.

Er empfielt uns den 'Tuzak-Weg', um in die Stadt zu kommen. Dorf Alrudan ist auf dem Weg.

Karypto will noch plaudern. 'Ihr könntet Fürst Herdin den Kopf abschlagen.' Die Haranidad sind sehr gut informiert, keine offenen Auseinandersetzungen. Diskus von Boron: eher direkte Angriffe.
Haranidad sind im Südosten von hier. Nach Tuzak: ca. eine Woche. Verpflegung für 2 Tage bis zu dem Dorf, Übernachtung heute, Kosten: 50 Dukaten. 

\stats{+4 LeP} Ruhephase, also \stats{22 LeP}. 
\paragraph{9. Praios 1019 BF}
\label{sec:9.-praios-1019}
Wir werden an den Rand des Gebietes geführt und bekommen eine Wegbeschreibung zum Pfad. Wir laufen stundenlang durch den Dschungel.

\mybox{8.5}{\page{Ich weis nicht, dass ich im Perlenmeer gesucht bin!}}

Da erkennen wir einen Holzunterstand des Mittelreichs mit einer mittelreichischen Wache. Er wirkt wahnsinnig. Als er uns bemerkt, fällt er von seinem Unterstand und bricht sich bei dem Fall sein Genick. 
 Während wir die Situation untersuchen, interessiert sich unsere Nandusgeweihte für die Flora des Waldes \dots \,Sie schreit. Wir rennen zu ihr, Sonnhard bleibt am Weg. Sie wird von einem Blatt eingerollt. Wir retten sie. 

Ich falle, beim Versuch, den Hochsitz zu erklimmen, herunter:
\stats{- 5 LeP}.
Wir bereiten uns auf die Nacht in einem Baum vor. Varna geht es immer noch schlecht. SPINNENANGRIFF!!
\stats{-3 LeP, also 14 LeP}
Regeneration
\stats{+ 4 LeP, also 18 LeP}.

\paragraph{10. Praios 1019 BF}

Wir sehen nach einem kurzen Fußmarsch verfallene Holzpalisaden; dazu verkommene Soldaten, die verrohrt, verwildert wirken. Gefangener Mittelländer sitzt in einem Käfig.
Wir beschließen, in mittelländischer Kleidung aufzutreten. Das Fort unter Fürst Herdin heißt \name{Retoglück} mit Kommandant \name{Trastan von Erlenholm}. Nach Aussage leben hier 34 Mann.
Corporal \name{Gerich Achsenbrecher}, 20 Jahre, Koscher Regiment.
\stats{-4 LeP} Messergras. Wir erreichen die Jergan-Tuzak-Straße und am Abend Alrudan, ungefähr tausend Einwohner. 'Normalität'. Maraskanische Gaststätte. Es gibt Kanitschaab zu essen, eine Art Tomatenspeise, süß gewürzt.

\paragraph{11. Praios 1019 BF}
\gold{-2 Silber} für die Übernachtung. Patroullie überrascht uns bei einer Pause.
Weitere Rebellengruppen:
\begin{enumerate}
\item \name{Mari Marasna} (übersetzt: Stachel der Maraske, Gifteinsatz, Zusammenarbeit mit dem berüchtigten maraskanischen Hexenzirkel \name{Rächerinnen Lycosas}, Gebiet im Norden der Insel),
\item \name{Octagonbanner} (Gruppe von übergelaufenen Mittelreichsoldaten),
\item noch einige andere, kleine Gruppen.
\end{enumerate}
Wir reisen weiter. Kampfschauplatz mit purpur-gelben und roten Pfeilen, ungefähr 2-3 Tage alt; rote Gruppe wurde anscheinend angegriffen. Abends sind wir geschafft und landen an einer Wegkreuzung.

\gold{+ 1 Bonus-AP} für gutes Rollenspiel.

\stats{+ 3 LeP, gesamt 15 LeP} Regeneration.

\paragraph{12. Praios 1019 BF}
Sieben gut gerüstete Soldaten marschieren nachts an unserem Nachtlagerbaum vorbei. Sie tragen ein Wappen, aber wir erkennen es nicht.

Wir reisen einen halben Tag, und sehen die Maraskankette und kommen an einen See. Wir werden angegriffen von einer Gruppe Maraskani, fünf Überfäller und zwei Fernkämpfer. Vogel krächzst 'Kriagarak - Kriagarak'. Geschosse sind purpur-gelb.

Neue Angreifer kommen. Es sind Maraskaner. 'Willkommen im Haranidad. Wir hoffen, ihr habt gute Gründe, so offen euer Leben zu riskieren. Der Haran wird über euch entscheiden'. Schwarze Armbinden.

Wir bekommen Augenbinden umgelegt und durch den Dschungel geführt. Lange. Es geht nach unten, es wird kühler. Stimmen. Wir landen in einer Höhle. Hüne, dunkle Haare, dick. Der Haran.

Er stellt uns Fragen über unsere Motivation, nach Maraskan zu reisen. Die Angreifer waren Mitglieder des Diskus von Haran.
König Dajin der Siebte ist der Herrscher, die Haranidad wohnen in einem wundersamen Tal. 
Wir werden in eine Zelle geführt und schlafen. \stats{+3 LeP, insgesamt 18 LeP}. Wir machen eine Teambesprechung. 

Wir bitten um ein Gespräch mit dem Haran. Seine Meinung:
\begin{enumerate}
\item Mittelländische Besatzung, Fürst Herdin: Harte Mittel, Überfall auf die Enduriummiene; nutzt Mittel, die dem Haran nicht gefallen. 
\item Große Erfolge, zum Beispiel mit der Enduriummiene. 
\item Sinoda wurde von der mittelreichischen Besatzung befreit. 
\item Wipfeltiger, Sira jerg Anack, Fren' Chairamarustazzim. 
\item 9 Stein Endurium, aber keine eigenen Alchimisten und Schmiede, es wurde an die Rur und Gror Kirche gespendet. 
\item Echsenstadt Akrabaal, 'Saurologe' Hilbert von Puspereiken ist in der Nähe, 'seltsamer Mittelreichler'. 
\item Widbrück unternimmt Schritte zur Isolation der Insel, nicht nur vorteilhaft für die Besatzung selbst. 
\item Will Maraskan von den Besatzern befreien.
\item Zweiter Enduriumüberfall war nicht von den Haranidad! Gerüchte: die Rebellengruppe Rurijidas Schwert, finstere Uljaykin (Erzfeinde!).
\item Seltsame, missgebildete Tiere und Pflanzen im Nebelwald an der Maraskankette (Saurologen fragen). 
\item Ich erzähle ihm von Borbarad, maraskanischer Name: \_\_\_\_\_\_\_\_\_\_\_\_\_\_ (NaN)
\item Tod der schwarzen Brazurgaar, Rand des Gebietes der Haranidad, Saurologe, Übergabe an die Urijidas Schwert, an der Grenze 
\end{enumerate}

Erkennungsmerkmale der Stämme:
\begin{enumerate}
\item Deijinin: {\color{red}Rot}
\item Diskus von Boran: {\color{yellow}gelb}-{\color{purple}Purpur}
\item Haranidad: schwarzes Armband
\item Uljaykin(die vom dunklen Tod): Diskus mit vier nach außen gerichteten Dolchen
\item Urijidas Schwert: {\color{purple}purpurnes} Armband mit {\color{yellow}gelben} Streifen
\end{enumerate}
Ch' Hl\^ar Wurzel färbt die Haut schwarz. Wir kriegen was davon. An der Oberfläche: Dorf, wenige Leute. 'Mächtige Haranidad?!?!' Leichenszene: Leiche wird angeschrieen. Er ist im Kampf mit der \name{schwarzen Brazurgaar}, Dschungelspinne. Höhle in der Nähe, Biss mit schnelle Lähmung, in der Dunkelheit überfällt sie ihre Gegner, spuckt ganzes Netz.

Haran hat vier Hartholzharnische, die er uns verkaufen kann ($\sim$ 150 Dukaten). 

\stats{+ 5 Lep, 23 LeP} Regeneration. 

\paragraph{13. Praios 1019 BF}
\label{sec:13.-praios-1019}
Regenerationstag. Die Haranidad halten Hühner. 
Weiße Würmer sind genießbar und sehr nährreich. \stats{+ 2 Lep, 25 LeP}, \stats{+ 7 Lep, 32 LeP} Juhuuu!! Wir halten Kriegsrat. 

\paragraph{14. Praios 1019 BF}
\label{sec:14.-praios-1019}
Die Jagd beginnt. Führerin \name{Alijida} bringt uns zu der Höhle. Die Netze brennen nicht gut, aber man kann sie zerschlagen. 

Den Wolfszahn präpariere ich mit Kugris. 

Ich bekomme eine Portion Antidot. 

\paragraph{Kampf gegen die schwarze Brazurgaar}
 Varna und ich plazieren uns in der Eingangsmulde. Ein Spinnennetzball kommt aus der Höhle. Nichts passiert. Wir werfen eine Fackel in die Höhle. Die Spinne, ein gewaltiges Wesen mit den Ausmaßen von drei Schritt in der Breite und zwei Schritt in der Höhe, hockt an der linken Seitenwand. Als Sonea zu uns herabkommen will, tritt sie versehentlich auf das Netz und ein weiterer Ball Spinnennetz schießt auf sie zu. Sie kann ihm nicht entweichen, sodass sie in das Netz gewickelt ist. Plötzlich schießt die Spinne aus der Höhle, und der Kampf entbrennt.  

-6 LeP in den Schwertarm, -3 LeP in den Schwertarm. 
Rest: 21 LeP

Armatrutz: \stats{+3 Rüstung} auf alle Körperteile

Kunstvolles Spinnennetz satuarischen Ursprungs. In der Mitte ist Frau mit einem schwarzen Spinnennetz auf der linken Wange. Mächtige Feindin \dots War die Spinne ihr Vertrautentier?

Wir kehren zurück in das Lager der Haranidad. 
\stats{+ 4 LeP, insg. 25 LeP} Regeneration. 

\paragraph{15. Praios 1019 BF}
\label{sec:15.-praios-1019}

Spinne ist seit ungefähr einem halben Jahr hier. Der Saurologe ist seit etwa einem halben Jahr hier, macht aber noch nicht so lange seine Ausgrabungen. Geschenk: Beutel mit getrockneten Früchten. Ernährung für eine halbe bis ganze Woche. Arajin: Farbhändler in Tuzak, Erkennungsformel 'Preiset den siebten König'. Führer bringt uns zum Saurologen. Libelle sticht mich. 

Abends kommen wir bei den Ausgrabungen an und treffen Hilbert von Puspereiken an, ca. 30 Jahre alt. \stats{- 1 LeP} Feuerassel

Ausgrabung im Norden: Ssel'Alth'Alch, lange gesuchtes Heiligtum, dort soll eines der elf Siegel zu Akrabaal sein. Drachengardisten unter der Führung von Praiotin von Rallerau haben herumgeschnüffelt und die Arbeit behindert, drei Schritt großer echsischer Siegelwächter wurde erweckt. In Sel'Ah'Pach könnte ein Siegel liegen. Neuer Büttelminister Delian von Widbrück. Borotin Almachios hat sechs Finger an beiden Händen. Alte Artefakte interessieren ihn.  
\stats{+ 2 LeP, insg. 27 LeP} Regeneration. 

\paragraph{16. Praios 1019 BF}
\label{sec:16.-praios-1019}
Richtung Südosten. Wir treffen mittags 11 Rebellen im Dschungel. Der Diskus von Boran hat sein Gebietsanspruch vergrößert. Maraskan-Federn unterbrechen unsere Nachtruhe, gelb-grüne Tausendfüßler, die sich an uns festsaugen. 

\paragraph{17. Praios 1019 BF}
\label{sec:17.-praios-1019}
Karypto entdeckt, dass sein Auge auch auf Borbarad reagiert. Er kann feststellen, ob Borbarad am seinem eigenen Aufenthaltsort war. Vogel lacht schadenfroh. Überfall. Schlinger taucht auf. Wir rennen. Drei weitere Rebellen sind noch bei uns. Wir überqueren den Roab. Wir finden eine bemooste, hölzerne Ruine. Innen erwarten uns drei Echsen.
\paragraph{Echsen}
\label{sec:echsen}

Sie haben gelbe Augen und rote, gelbe Schuppen. Sie sagen 'Wir haben euch erwartet' und erzählen ihr Anliegen: 

\begin{enumerate}
\item Sie sind Ausgesandte von Akrabaal, die Echsenstadt, die durch elf starke Siegel gestärkt und geschützt wird. Das Siegel \name{Tzel'Al'Tach} ist in Feindeshände geraten: es ist ein altes Szepter (\name{Crrzassh}). Es wurde vor einem Jahr gestohlen, und die Echsen vermuten, dass ein 'Freund', '\name{Tiuch'zi'teuy}' der Echsen geholfen hat, es zu klauen. Entsprechender Diener hat finstere, große Macht und das Szepter wurde missbraucht. Da es Menschen waren, die es geklaut haben, sollen auch Menschen (wir) es wieder zurückbringen. 
\item Das Szepter besteht aus schwarzem Holz, durchzogenen goldene Fäden. Es ist 60 Finger lang und hat eine ein Spann große Kugel, die goldgeädert ist. 
\item Das verderbte Tal ist das Heim der Skretchu, einer alten Echsenschlange, das sollen wir meiden. Sie hasst Menschen. 
\item Wenn wir Hilfe brauchen, sollen wir die Echsen über die Libelle rufen.
\item Eine der Echsen heißt \name{Chichim}. 
\item Wir zeigen ihnen die Al'Anfanischen Prophezeiungen. 
  \begin{enumerate}
  \item  Sie erzählen uns noch, dass sie Pyrdacor als den 'geraubte Schlangenfürst' aus den Al'Anfaner Prophezeihungen kennen. 
\item Ich kann mit ihnen per Gedanken, also drachisch, reden. Darüber erfahre ich, dass 'das kühne Tier mit dem Krötensinn' bei ihnen eine Kröten-Wächterstatue der Siegel bezeichnen. 
  \end{enumerate}
\item Als Belohnung geben sie uns vorab ein Beutelchen mit Edelsteinen mit einem Gegenwert von \gold{+120 Dukaten} pro Person. 
\item Die Hexe \name{Laranja} gehört zu der schwarzen Brazurgaar. 
\end{enumerate}

\stats{+ 2 LeP, insg. 29 LeP} Regeneration. 
\paragraph{18. Praios 1019 BF}

Aktive Aufträge:
\begin{enumerate}
\item {\color{green}Von uns selbst: Weltgefüge retten}
\item {\color{red}Delian von Widbrück: Kontaktieren, mit ihm reden}
\item {\color{green}Boron-Kirche: den Überfall auf die Enduriummiene untersuchen, eventuell das Endurium zurückbringen.}
\item {\color{red}Wipfeltiger: Kopf von Fürst Herdin}
\item {\color{orange}Hexenzirkel: Gefahr!}
\item {\color{green}Echsen: Szepter finden, langfristig Borbarad aufhalten.}
\end{enumerate}
Wir reisen weiter. Nach zwei Stunden verlassen uns unsere Führer. Wir kommen in das Gebiet der Uljaikin. Die Unannehmlichkeiten des Dschungels werden uns bewusst. 
Abends finden wir eine Höhle zum Schlafen. Wir vernichten Jagdgras. 
\stats{+ 3 LeP, insg. 32 LeP} Regeneration. 

\paragraph{19. Praios 1019 BF}
Nebel. Gruppe von etwa 20 Personen von den Ulyaykin. Gegen 100 Dukaten dürfen wir das Gebiet passieren. Der Name des Sprechers ist \name{Dajin Manchial}
\paragraph{20. Praios 1019 BF} 
\stats{-4 LeP, insg. 28 LeP} über den Tag. Wir kommen an der Mine an. Keller, alles heftig umkämpft. Wachtürme mit Leichen, Kasernen, Stall mit Scheune, Kräuter und Gemüsegärten. 

Was im Logbuch steht:
Im Boronsmond des letzten Jahr sollte die Ablösung des 4. Banners schon dagewesen sein, um das 3. Banner abzulösen. Sie kamen viel später als geplant, Anführer ist KGIA-Oberst Praiotin von Rallerau. Die Aufrührer der Aufstände, die aufgrund der späten Ablöse entstanden sind, werden brutal von diesem bestraft. 
Schließlich erwirkt er mit einem Schreiben von Fürst Herdin und Delian von Widbrück, dass das 4. Banner und er den Abtransport des Enduriums durchführen. Das 3. Banner muss also an der Mine bleiben, die Frauen und Männer sind sehr deprimiert und verzweifelt. Es wird der Befehl erlassen, keine zusätzliche Bewachung vorzunehmen, auch wenn der Kommandant das für sinnvoll hält. Dann brechen die Eintragungen ab. 

Zwei Marasken, rot-gelbe Spinnen, hausen in den Leichen und ich spüre sie auf \dots \stats{-2 LeP} durch Biss \stats{-2 LeP} auf den Arm. Rest \stats{24 LeP}

Ich sammle das Gift der Marasken ein, eine Portion kann ich extrahieren. 
\mybox{8}{Maraskengift: \\
Nahkampfwaffe: Bei Treffer 4 KR 1W6 Schaden\\
Pfeile (5): 1W6 Schaden\\
gegen eine um 4 erschwerte KO Probe}

In der Küche finden wir zwei tote Paktierer (Mittelländer und Tulamide): Belharhar (Rondra) und Bel'Zorash (Firun), außerdem viel Ausrüstung mittelreichischer Soldaten, auch maraskanische Bekleidung, Expeditionsausrüstung. 

\paragraph{In der Mine}
\label{sec:der-mine}

15 Schritt breiter Gang, nach 15 Schritt Kreuzung, Gang nach links und rechts. Diese führen zu zwei Räumen in denen jeweils 2-3 zerstückelte Soldaten, von kräftigen Klauen zerrissen, ein Zant?!, neben eine Balliste liegen, in den Wänden sind Schießscharten. Die Wände sind pechschwarz, Endurium färbt die Umgebung schwarz. Den großen Gang weiter kommen wir in eine große Höhle, aus der eine Plattform mit drei Schritt Durchmesser nach unten führt, sonst gibt es keinen anderen Ausgang. Ein Bannkreis liegt auf dem Aufzug, mit sieben Zeichen.

Wir schweben per Seilbahn nach unten.
 
1. Untergeschoss: Großer Raum mit Bergen von Leichen von Minensklaven und maraskanische Wiederständlern (Ulyaykin), Heptagramm.

3. Untergeschoss: Große Hitze, Weißes Gespinst, Schleim, Säure, interessant, da in letzter Zeit benutzt. Wimmernder. 

Viergehörnter Achor Hobai, der weiße Wurm, der große Alchimist (kein Ingame-Wissen!!), tot,

Wir finden wieder aus der Mine.

\mybox{7}{\gold{+ 350 AP, *Klettern, *Pflanzenkunde, *Gesteinskunde, *Überreden, *HK Gift, *Kriegskunst, *Athletik, *Sich Verstecken, *Alchimie}}

Stand: \stats{24/32 LeP}

%%% Local Variables:
%%% mode: latex
%%% TeX-master: "borb_head"
%%% End: